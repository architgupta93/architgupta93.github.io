\begin{tabularx}{0.95\textwidth}{lr}
\textbf{Music and Brain: Perception of Rhythm} &
\href{https://www.ee.iitb.ac.in/web/faculty/homepage/bipin}{Prof. Bipin
Rajendran} \\
Neuromorphic Engineering & Autumn 2014-2015 \\
\multicolumn{2}{p{16cm}}{
\begin{itemize}
        \item Neurons, for computational simplicity, are usually modelled as
Integrators. However, several properties of the biological neuron, like
resonance and ion-channel dynamics, cannot be expressed by a simple integrator
model.
        \item We demonstrated that perception of music is closely associated
with these properties by using a Resonate and Fire model(Izhikevich [2001]) for
beat tracking. Our results closely matched the response of human subjects who
were asked to tap along with the music.
\end{itemize}
}\\

\textbf{Load analysis and energy efficient operation of  Cellular Networks} &
\textsc{July 2014 - Dec. 2014}\\
\href{https://www.ee.iitb.ac.in/~infonet/}{InfoNet Laboratory} & \href{https://www.ee.iitb.ac.in/wiki/faculty/karandi}{Prof. Abhay Karandikar}\\
\multicolumn{2}{p{16cm}}{
\begin{itemize}
	\item Base Stations (BS) are set up to meet the peak Quality of Service
(QoS) demands in a locality. However, operational costs can be drastically cut
down by reorganizing the network dynamically. We developed a model to predict
the network state (Voice traffic at each BS) by analytically modeling the traffic 
data of one of India's leading telecom operators. 
	\item We also demonstrated the feasibility of saving operational cost
using a simple ON/OFF scheme, where we determine the optimal number of BSs
required to maintain coverage. Our \href{http://www.ee.iitb.ac.in/student/~architgupta93/projects/desc/btp-paper.htm}{paper} has been accepted at  the National Conference on Communications, 2015.
\end{itemize}
}\\
 


\end{tabularx}
